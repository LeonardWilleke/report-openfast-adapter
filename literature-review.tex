\section{Literature Review}
\label{section:review}

A recent advancement in the coupling of OpenFAST is the tool \textbf{AspFAST} \cite{Taschner:2022}. Developed at the Delf University of Technology and the weather forecasting company Whiffle, it connects OpenFAST to the closed-license Large Eddy Simulation (LES) code GRASP. 
As a binder code, AspFAST serves as an intermediary between the GPU-driven GRASP and the CPU-driven OpenFAST. It accesses OpenFAST through a C++ API, essentially operating as a driver code. GRASP is accessed through a specialized API that handles the transition from CPU to GPU operations.
The core functionalities of AspFAST include the data exchange and mapping between GRASP and OpenFAST. This is done for crucial data such as force, velocity, and the position of turbine blades. Notably, the code implements the mapping between the volume-based velocity grid of GRASP and the actuator line model of OpenFAST. Because the coupling  code\footnote{\url{https://gitlab.com/whiffle-public/aspfast} (visited on 20.12.2023)} is open-source, it can serve as a blueprint for anyone interested in coupling OpenFAST via the C++ interface.
The software architecture of AspFAST is not far from our idea for the preCICE-OpenFAST adapter: AspFAST is a plug-in for the CFD solver GRASP while being the driver code for OpenFAST, serving as the "adapter" between the two. A big difference lies in the fact that AspFAST also has to implement the mapping algorithms, which are executed by preCICE in our approach.

The \textbf{ExaWind} project \cite{Sprague:2019} is based on the open-source, massively paralles and incompressible flow solver NaluWind \cite{Ananthan:2019}. It is a wind-focused adaption of the general flow solver NaluCFD and strives to deliver flow simulations in exascale. Hence, great emphasis lies on the scalability and simulation speed of the software through advanced solver and coupling algorihtms as well as the use of GPUs. The coupling to OpenFAST is implemented inside NaluWind and not as additional plug-in. An extensive documentation with insights into the theory and verification is given. 

Another coupling based on solver modification is the Simulator for Wind Farm Applications or \textbf{SOWFA} \cite{Churchfield:2012} from NREL. It couples OpenFOAM to OpenFAST by providing an additional  class \textit{horizontalAxisALM} in OpenFOAM. It defines functions to call OpenFAST and enables the data exchange. This class can be included directly in any OpenFOAM solver by modifying the OpenFOAM source code, so that the solver calls OpenFAST during each time step.

A software solution relying on a similar coupling tool to preCICE is \textbf{foamFAST} \cite{Weber:2017}. This project uses the commercial coupling tool for partitioned multi-physics simulations MpCCI, developed and sold by the Fraunhofer SCAI. Similar to preCICE, MpCCI is connected to different programs via adapters. Once the adapter is written, you can couple your tool to any other tool in the MpCCI environment. A difference in software architecture is the fact that MpCCI acts as the master algorithm: It calls the solvers instead of being called by them like preCICE. During the project, a MpCCI adapter for OpenFAST was developed and an existing OpenFOAM adapter modified to perform the coupling between the two.\\

The reviewed literature shows two different approaches on how to undertake the coupling. The first approach is to \textbf{modify the CFD solver source code}, as the ExaWind project has done with the NaluWind solver and SOWFA with OpenFOAM. While convenient for developers who are familiar with the software, this approach can prove hard to maintain when the underlying solver software is updated.

The second approach leaves the CFD solvers intact and uses \textbf{plug-ins or master algorithms} to connect the software tools. The plug-in approach achieves the coupling via an additional module or library that can be called by the original CFD solver. An example is AspFAST, where the stand-alone plug-in code is maintained seperately from the solver. foamFAST implements a master algorithm and is especially interesting because the coupling tool MpCCI it relies on is similar to preCICE.

Given the previous work, why should we continue to develop a preCICE-OpenFAST adapter?
Our approach combines some of the positive aspects of previous projects. First, using preCICE allows to rely on advanced and tested coupling algorithms without the need to re-implement them or buy a licensed software. Establishing a preCICE-OpenFAST adapter opens up the road to connect OpenFAST not only to OpenFOAM, but to any CFD solver coupled to preCICE. Furthermore, preCICE creates a clean and easy to use simulation environment and simplifies setting up and maintaining simulations.

Nevertheless, it should be noted that OpenFOAM is currently the only CFD solver connected to preCICE which can be used for a coupling with OpenFAST. Coupling any other tool, such as the presented GRASP or NaluCFD, will require the development of additional adapters.

\begin{comment}
% Not relevant for our case
\textbf{OF2: A coupling library for OpenFOAM}
\begin{itemize}
\item Motivation: Model FOWTs with mixed fidelity to speed up the design process
\item OpenFOAM computes the floater hydrodynamics with high fidelity
\item OpenFAST computes the rotor aerodynamics and servo-elastic response in low-fidelity
\item Method: Two OpenFOAM libraries perform the coupling
\item libForcedOpenFAST: Software layer to interact with and execute OpenFAST
\item libOF2: Software layer to read and write data from OpenFOAM
\item Both libraries communicate with each other
\item Benefits: Less computationally expensive than CFD only, simulation of large turbine displacements (drift, rotation) possible
\item Drawbacks: Done for hydrodynamics (OpenFOAM replaces HydroDyn), not sure if this approach would work for aerodynamics as well (yes)
\\
\end{itemize}

% Used
\textbf{AspFAST} \cite{Taschner:2022}
\begin{itemize}
\item Motivation
\begin{itemize}
\item Wind farm simulation combines the analysis of large-scale flow dynamics with individual turbine behavior
\item Different tools are needed to model these two phenomena
\item AspFAST couples the LES code GRASP with the wind turbine tool OpenFAST to perform such simulations
\end{itemize}
\item Methodology
\begin{itemize}
\item AspFAST is a binder code between the GPU-driven GRASP and the CPU-driven OpenFAST
\item OpenFAST is accessed via C++ API → AspFAST is the driver code
\item AspFAST takes care of the mapping and communication between the tools
\end{itemize}
\item What AspFAST does
\begin{itemize}
\item Synchronize GRASP and OpenFAST
\item Exchange data: Force, Velocity, Position (of the turbine blades)
\item Map data and make use of the actuator line model
\begin{itemize}
\item Map velocity from GRASP to OpenFAST
\item Map force from OpenFAST to GRASP
\item Make use of both internal meshes inside of OpenFAST
\end{itemize}
\end{itemize}
\item How it relates to our idea
\begin{itemize}
\item Very close to what we want to achieve
\item Difference: GRASP is a commercial and licensed software, AspFAST allows coupling only to this one solver
\item But: AspFAST itself is open-source and can serve as a blueprint
\item The coupling of the parallel flow solver NaluWind with OpenFAST pursues a very similar idea
\\
\end{itemize}
\end{itemize}

\textbf{ExaWind: A Coupling of NaluWind with OpenFAST} \cite{Sprague:2019}\\
\begin{itemize}
\item NaluWind is an open-source massively parallel, incompressible flow solver for wind turbines and parks
\item Is a wind-focused adaption of the general flow solver NaluCFD
\item Focus on delivering flow simulations in exascale
\item Coupled to OpenFAST to perform multi-fidelity simulations of wind parks
\item Verified simulation on single turbine case against other codes and experiments
\item Emphasis on scalability and increased simulation speed through advanced solver and coupling algorithms and the optional use of GPUs
\item Extensive documentation with insights into the theory and verification 
\item How does it relate to our work
\begin{itemize}
\item Very close to what we want to do
\item Especially interesting for high-performance applications
\item Expert tool that needs some time to set up, understand, and run --> The hope is that our coupling to the "standard" tool OpenFOAM will have a lower barrier for beginners
\end{itemize}
\end{itemize}

\textbf{SOWFA} \cite{Churchfield:2012}
\begin{itemize}
\item Developed by NREL
\item Simulator for Wind Farm Applications based on OpenFOAM
\item Enables a coupling with OpenFAST
\item Methodology
\begin{itemize}
\item Implements a turbine class horizontalAxisALM in OpenFOAM to call OpenFAST and exchange data
\item This class can be included in any OpenFOAM solver
\item The solver calls OpenFAST during each timestep
\end{itemize}
\item How this relates to our idea
\begin{itemize}
\item Different approach: Is based on the modification of OpenFOAM source code while we strive to leave the solvers untouched
\item A lot of internal OpenFOAM knowledge necessary
\\
\end{itemize}
\end{itemize}

\textbf{foamFAST and MPCCI coupling tool} \cite{Weber:2017}
\begin{itemize}
\item Motivation
\begin{itemize}
\item Replace the low-fidelity aerodynamic calculations in FAST, based on the BEM theory, with high-fidelity calculations in OpenFOAM
\item BEM theory is not sufficient for many applications; CFD is needed
\item Use the coupling tool MpCCI to perform the coupling
\end{itemize}
\item Methodology
\begin{itemize}
\item Couple OpenFOAM and OpenFAST via the coupling tool MpCCI
\item MpCCI is a licensed coupling tool for partitioned multi-physics simulations developed by the Fraunhofer SCAI
\begin{itemize}
\item Connect tools to MpCCI via adapters
\item Once the adapter is written, you can couple your tool to any other tool connected to MpCCI
\item MpCCI acts as the master algorithm
\end{itemize}
\item This project developed an OpenFAST adapter from scratch and adapted an existing OpenFOAM adapter to perform the coupling
\item Source code is not violated
\end{itemize}
\item How it relates to our idea
\begin{itemize}
\item Very similar approach to preCICE´
\item Difference of MpCCI and preCICE: preCICE takes the library approach, while MpCCI takes the master algorithm approach
\item MpCCI is commercial and has limited HPC capabilities
\\
\end{itemize}
\end{itemize}

\begin{itemize}
\item Combines positive aspects of previous works while avoiding some drawbacks
\item Maintainability: Adapter is a separated piece of code which is easier to maintain than a modified OpenFOAM solver
\item Open access: No license
\item Plug and play: Adapter opens the road to connect OpenFAST not only to OpenFOAM, but to any CFD solver coupled to preCICE
\item Easy to use: preCICE creates a clean simulation environment, seperating the coupled tools clearly, from which students and professionals benefit
\end{itemize}
What are some serious drawbacks?
\begin{itemize}
\item There are currently no preCICE adapters for other CFD solvers that would be interesting to couple to: NaluWind, GRASP, ...
\item Developing such adapters would mean to re-implement a functionality that already exists in native coupling tools --> questionable if that is a good investment
\end{itemize}

\end{comment}