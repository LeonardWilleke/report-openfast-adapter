\documentclass{layout/tudelft-aiaa}

%% Additional packages and commands
\renewcommand{\deg}{\si{\degree}\xspace}
\usepackage[newfloat]{minted} % (recommended)
\usepackage{comment}
\usepackage{biblatex}
\addbibresource{article.bib}
% Set global Minted options
\setminted{linenos, autogobble, frame=lines, framesep=2mm}

%% Defining title, author and affiliation
\title{A preCICE-OpenFAST adapter \\ to couple OpenFAST to CFD simulation tools \\ \vspace{4pt} \normalsize{Technical Report}}
\author{Leonard Willeke}
\affil{Delft University of Technology}

%%%%%%%%%%%%%%%%%%%%%%%%%%%%%
%%%%% Begin of document %%%%%
%%%%%%%%%%%%%%%%%%%%%%%%%%%%%

\begin{document}

\AlwaysPagewidth{

\maketitle

%% Abstract

\begin{abstract}
    \noindent
    Flow dynamics in wind parks are important for many engineering problems, ranging from optimized control to load reduction. This work introduces a preCICE-OpenFAST adapter designed to seamlessly couple OpenFAST with the preCICE library. preCICE facilitates the black-box coupling of simulation programs, enabling multi-physics simulations. The adapter acts as a driver code for OpenFAST, steering the turbine simulation and utilizing preCICE for efficient data exchange with the flow simulation. In a proof of concept, the adapter is employed to couple OpenFAST with OpenFOAM, demonstrating its potential despite limited capabilities. The presented approach lays the foundation for more comprehensive wind park simulations by integrating OpenFAST in the preCICE ecosystem. This research contributes to advancing wind park modeling techniques, providing a framework for studying complex interactions within wind farms.
\end{abstract}
}

%% Nomenclature

\renewcommand{\nompreamble}{\emph{Use this section to provide a list of all symbols used in the report. Provide a concise description. For example:}} %

\printnomenclature[\nomequalsign]

\nomenclature{$A$}{amplitude of oscillation}
\nomenclature{$C_p$}{pressure coefficient}
\nomenclature{$C_x$}{force coefficient in the x direction}
\nomenclature{$C_y$}{force coefficient in the y direction}
\nomenclature{$c$}{chord}
\nomenclature{$dt$}{time step}
\nomenclature{$F_x$}{X component of the resultant pressure force acting on the vehicle}
\nomenclature{$F_y$}{Y component of the resultant pressure force acting on the vehicle}
\nomenclature{$f, g$}{generic functions}
\nomenclature{$h$}{height}
\nomenclature{$i$}{time index during navigation}

%% Main body

\section{Introduction}

The big picture
\\
 \begin{itemize}
 	\item Understanding the flow in wind parks can have a valuable impact on many fields
 	\begin{itemize}
 		\item Engineering: Optimize turbine design and control to minimize load while maximizing power output
 		\item Economics: Assess the economic viability
 		\item Environment and Policy: Impact on the local environment and how to regulate it
 	\end{itemize}
 	\item While the detailed, high-fidelity simulation of turbines is computationally very expensive, low-fidelity engineering models present an alternative
 	\item Model the physical phenomena in a simplified way to represent the most important aspects for the problem at hand
 	\item A popular engineering model for wind turbines is the Actuator Line Method
 	\item Turbine tower and blades are represented as slender beams
 	\item Allows the detailed study of wake effects and blade deformations
 	\item Open-source tool OpenFAST implements such models
 	\item Coupling OpenFAST with a high-fidelity LES CFD solver would allow to accurately solve the wind field at moderate computational cost \ref{fig:openfast:coupling}
 	\item preCICE is a multi-physics coupling library to couple different solvers
 	\item Coupled tools include OpenFOAM, an open-source CFD solver used in industry and academia
 	\item To couple a tool via preCICE, an additional piece of software called adapter is necessary
 	\item Idea: Write a preCICE-OpenFAST adapter to couple OpenFAST to CFD solvers via preCICE
 	\\
 \end{itemize}

\begin{figure*}[h]
	\centering
	\includegraphics[width=0.9\textwidth]{images/openfast-coupling-scheme.png}
	\caption{Coupling of OpenFAST to a CFD program. The fluid solver computes the flow dynamics and sends velocity data to OpenFAST. OpenFAST simulates the turbine dynamics and sends back forces and deflections, which are imposed on the flow field. Source: OpenFAST documentation\protect\footnotemark}
	\label{fig:openfast:coupling}
\end{figure*}

\footnotetext{{\url{https://ganesh-openfast.readthedocs.io/en/latest/_images/actuatorLine_illustrationViz.pdf}(visited on 07.11.2023)}}

Literature Review: Understand what has been done already to couple OpenFAST with CFD solvers and how our approach relates to that
\\
\begin{comment}
	

\textbf{OF2: A coupling library for OpenFOAM}
\begin{itemize}
    \item Motivation: Model FOWTs with mixed fidelity to speed up the design process
    \item OpenFOAM computes the floater hydrodynamics with high fidelity
    \item OpenFAST computes the rotor aerodynamics and servo-elastic response in low-fidelity
    \item Method: Two OpenFOAM libraries perform the coupling
    \item libForcedOpenFAST: Software layer to interact with and execute OpenFAST
    \item libOF2: Software layer to read and write data from OpenFOAM
    \item Both libraries communicate with each other
    \item Benefits: Less computationally expensive than CFD only, simulation of large turbine displacements (drift, rotation) possible
    \item Drawbacks: Done for hydrodynamics (OpenFOAM replaces HydroDyn), not sure if this approach would work for aerodynamics as well (yes)
    \\
\end{itemize}

\end{comment}

\textbf{AspFAST} \cite{Taschner:2022}
\begin{itemize}
	\item Motivation
	\begin{itemize}
		\item Wind farm simulation combines the analysis of large-scale flow dynamics with individual turbine behavior
		\item Different tools are needed to model these two phenomena
		\item AspFAST couples the LES code GRASP with the wind turbine tool OpenFAST to perform such simulations
	\end{itemize}
	\item Methodology
	\begin{itemize}
		\item AspFAST is a binder code between the GPU-driven GRASP and the CPU-driven OpenFAST
		\item OpenFAST is accessed via C++ API → AspFAST is the driver code
		\item AspFAST takes care of the mapping and communication between the tools
	\end{itemize}
	\item What AspFAST does
	\begin{itemize}
		\item Synchronize GRASP and OpenFAST
		\item Exchange data: Force, Velocity, Position (of the turbine blades)
		\item Map data and make use of the actuator line model
		\begin{itemize}
			\item Map velocity from GRASP to OpenFAST
			\item Map force from OpenFAST to GRASP
			\item Make use of both internal meshes inside of OpenFAST
		\end{itemize}
	\end{itemize}
	\item How it relates to our idea
	\begin{itemize}
		\item Very close to what we want to achieve
		\item Difference: GRASP is a commercial and licensed software, AspFAST allows coupling only to this one solver
		\item The coupling of the parallel flow solver NaluWind with OpenFAST pursues a very similar idea
		\\
	\end{itemize}
\end{itemize}


\textbf{foamFAST and MPCCI coupling tool} \cite{Weber:2017}
\begin{itemize}
	\item Motivation
	\begin{itemize}
		\item Replace the low-fidelity aerodynamic calculations in FAST, based on the BEM theory, with high-fidelity calculations in OpenFOAM
		\item BEM theory is not sufficient for many applications; CFD is needed
		\item Use the coupling tool MpCCI to perform the coupling
	\end{itemize}
	\item Methodology
	\begin{itemize}
		\item Couple OpenFOAM and OpenFAST via the coupling tool MpCCI
		\item MpCCI is a licensed coupling tool for partitioned multi-physics simulations developed by the Fraunhofer SCAI
		\begin{itemize}
			\item Connect tools to MpCCI via adapters
			\item Once the adapter is written, you can couple your tool to any other tool connected to MpCCI
			\item MpCCI acts as the master algorithm
		\end{itemize}
		\item This project developed an OpenFAST adapter from scratch and adapted an existing OpenFOAM adapter to perform the coupling
		\item Source code is not violated
	\end{itemize}
	\item How it relates to our idea
	\begin{itemize}
		\item Very similar approach to preCICE´
		\item Difference of MpCCI and preCICE: preCICE takes the library approach, while MpCCI takes the master algorithm approach
		\item MpCCI is commercial and has limited HPC capabilities
		\\
	\end{itemize}
\end{itemize}


\textbf{SOWFA} \cite{Churchfield:2012}
\begin{itemize}
	\item Developed by NREL
	\item Simulator for Wind Farm Applications based on OpenFOAM
	\item Enables a coupling with OpenFAST
	\item Methodology
	\begin{itemize}
		\item Implements a turbine class horizontalAxisALM in OpenFOAM to call OpenFAST and exchange data
		\item This class can be included in any OpenFOAM solver
		\item The solver calls OpenFAST during each timestep
	\end{itemize}
	\item How this relates to our idea
	\begin{itemize}
		\item Different approach: Is based on the modification of OpenFOAM source code while we strive to leave the solvers untouched
		\item A lot of internal OpenFOAM knowledge necessary
		\\
	\end{itemize}
\end{itemize}

\textbf{ExaWind: A Coupling of NaluWind with OpenFAST} \cite{Sprague:2019}\\
\begin{itemize}
	\item NaluWind is an open-source massively parallel, incompressible flow solver for wind turbines and parks
	\item Is a wind-focused adaption of the general flow solver NaluCFD
	\item Focus on delivering flow simulations in exascale
	\item Coupled to OpenFAST to perform multi-fidelity simulations of wind parks
	\item Verified simulation on single turbine case against other codes and experiments
	\item Emphasis on scalability and increased simulation speed through advanced solver and coupling algorithms and the optional use of GPUs
	\item Extensive documentation with insights into the theory and verification 
	\item How does it relate to our work
	\begin{itemize}
		\item Very close to what we want to do
		\item Especially interesting for high-performance applications
		\item Expert tool that needs some time to set up, understand, and run --> The hope is that our coupling to the "standard" tool OpenFOAM will have a lower barrier for beginners
	\end{itemize}
\end{itemize}

Given the previous work, why should we continue to develop a preCICE-OpenFAST adapter?
\begin{itemize}
	\item Combines positive aspects of previous works while avoiding some drawbacks
	\item Maintainability: Adapter is a separated piece of code which is easier to maintain than a modified OpenFOAM solver
	\item Open access: No license
	\item Plug and play: Adapter opens the road to connect OpenFAST not only to OpenFOAM, but to any CFD solver coupled to preCICE
	\item Easy to use: preCICE creates a clean simulation environment, seperating the coupled tools clearly, from which students and professionals benefit
\end{itemize}
What are some serious drawbacks?
\begin{itemize}
	\item There are currently no preCICE adapters for other CFD solvers that would be interesting to couple to: NaluWind, GRASP, ...
	\item Developing such adapters would mean to re-implement a functionality that already exists in native coupling tools --> questionable if that is a good investment
\end{itemize}


\section{Literature Review}
\label{section:review}

A recent advancement in the coupling of OpenFAST is the tool \textbf{AspFAST} \cite{Taschner:2022}. Developed at the Delf University of Technology and the weather forecasting company Whiffle, it connects OpenFAST to the closed-license Large Eddy Simulation (LES) code GRASP. 
As a binder code, AspFAST serves as an intermediary between the GPU-driven GRASP and the CPU-driven OpenFAST. It accesses OpenFAST through a C++ API, essentially operating as a driver code. GRASP is accessed through a specialized API that handles the transition from CPU to GPU operations.
The core functionalities of AspFAST include the data exchange and mapping between GRASP and OpenFAST. This is done for crucial data such as force, velocity, and the position of turbine blades. Notably, the code implements the mapping between the volume-based velocity grid of GRASP and the actuator line model of OpenFAST. Because the coupling  code\footnote{\url{https://gitlab.com/whiffle-public/aspfast} (visited on 20.12.2023)} is open-source, it can serve as a blueprint for anyone interested in coupling OpenFAST via the C++ interface.
The software architecture of AspFAST is not far from our idea for the preCICE-OpenFAST adapter: AspFAST is a plug-in for the CFD solver GRASP while being the driver code for OpenFAST, serving as the "adapter" between the two. A big difference lies in the fact that AspFAST also has to implement the mapping algorithms, which are executed by preCICE in our approach.

The \textbf{ExaWind} project \cite{Sprague:2019} is based on the open-source, massively paralles and incompressible flow solver NaluWind \cite{Ananthan:2019}. It is a wind-focused adaption of the general flow solver NaluCFD and strives to deliver flow simulations in exascale. Hence, great emphasis lies on the scalability and simulation speed of the software through advanced solver and coupling algorihtms as well as the use of GPUs. The coupling to OpenFAST is implemented inside NaluWind and not as additional plug-in. An extensive documentation with insights into the theory and verification is given. 

Another coupling based on solver modification is the Simulator for Wind Farm Applications or \textbf{SOWFA} \cite{Churchfield:2012} from NREL. It couples OpenFOAM to OpenFAST by providing an additional  class \textit{horizontalAxisALM} in OpenFOAM. It defines functions to call OpenFAST and enables the data exchange. This class can be included directly in any OpenFOAM solver by modifying the OpenFOAM source code, so that the solver calls OpenFAST during each time step.

A software solution relying on a similar coupling tool to preCICE is \textbf{foamFAST} \cite{Weber:2017}. This project uses the commercial coupling tool for partitioned multi-physics simulations MpCCI, developed and sold by the Fraunhofer SCAI. Similar to preCICE, MpCCI is connected to different programs via adapters. Once the adapter is written, you can couple your tool to any other tool in the MpCCI environment. A difference in software architecture is the fact that MpCCI acts as the master algorithm: It calls the solvers instead of being called by them like preCICE. During the project, a MpCCI adapter for OpenFAST was developed and an existing OpenFOAM adapter modified to perform the coupling between the two.\\

The reviewed literature shows two different approaches on how to undertake the coupling. The first approach is to \textbf{modify the CFD solver source code}, as the ExaWind project has done with the NaluWind solver and SOWFA with OpenFOAM. While convenient for developers who are familiar with the software, this approach can prove hard to maintain when the underlying solver software is updated.

The second approach leaves the CFD solvers intact and uses \textbf{plug-ins or master algorithms} to connect the software tools. The plug-in approach achieves the coupling via an additional module or library that can be called by the original CFD solver. An example is AspFAST, where the stand-alone plug-in code is maintained seperately from the solver. foamFAST implements a master algorithm and is especially interesting because the coupling tool MpCCI it relies on is similar to preCICE.

Given the previous work, why should we continue to develop a preCICE-OpenFAST adapter?
Our approach combines some of the positive aspects of previous projects. First, using preCICE allows to rely on advanced and tested coupling algorithms without the need to re-implement them or buy a licensed software. Establishing a preCICE-OpenFAST adapter opens up the road to connect OpenFAST not only to OpenFOAM, but to any CFD solver coupled to preCICE. Furthermore, preCICE creates a clean and easy to use simulation environment and simplifies setting up and maintaining simulations.

Nevertheless, it should be noted that OpenFOAM is currently the only CFD solver connected to preCICE which can be used for a coupling with OpenFAST. Coupling any other tool, such as the presented GRASP or NaluCFD, will require the development of additional adapters.

\begin{comment}
% Not relevant for our case
\textbf{OF2: A coupling library for OpenFOAM}
\begin{itemize}
\item Motivation: Model FOWTs with mixed fidelity to speed up the design process
\item OpenFOAM computes the floater hydrodynamics with high fidelity
\item OpenFAST computes the rotor aerodynamics and servo-elastic response in low-fidelity
\item Method: Two OpenFOAM libraries perform the coupling
\item libForcedOpenFAST: Software layer to interact with and execute OpenFAST
\item libOF2: Software layer to read and write data from OpenFOAM
\item Both libraries communicate with each other
\item Benefits: Less computationally expensive than CFD only, simulation of large turbine displacements (drift, rotation) possible
\item Drawbacks: Done for hydrodynamics (OpenFOAM replaces HydroDyn), not sure if this approach would work for aerodynamics as well (yes)
\\
\end{itemize}

% Used
\textbf{AspFAST} \cite{Taschner:2022}
\begin{itemize}
\item Motivation
\begin{itemize}
\item Wind farm simulation combines the analysis of large-scale flow dynamics with individual turbine behavior
\item Different tools are needed to model these two phenomena
\item AspFAST couples the LES code GRASP with the wind turbine tool OpenFAST to perform such simulations
\end{itemize}
\item Methodology
\begin{itemize}
\item AspFAST is a binder code between the GPU-driven GRASP and the CPU-driven OpenFAST
\item OpenFAST is accessed via C++ API → AspFAST is the driver code
\item AspFAST takes care of the mapping and communication between the tools
\end{itemize}
\item What AspFAST does
\begin{itemize}
\item Synchronize GRASP and OpenFAST
\item Exchange data: Force, Velocity, Position (of the turbine blades)
\item Map data and make use of the actuator line model
\begin{itemize}
\item Map velocity from GRASP to OpenFAST
\item Map force from OpenFAST to GRASP
\item Make use of both internal meshes inside of OpenFAST
\end{itemize}
\end{itemize}
\item How it relates to our idea
\begin{itemize}
\item Very close to what we want to achieve
\item Difference: GRASP is a commercial and licensed software, AspFAST allows coupling only to this one solver
\item But: AspFAST itself is open-source and can serve as a blueprint
\item The coupling of the parallel flow solver NaluWind with OpenFAST pursues a very similar idea
\\
\end{itemize}
\end{itemize}

\textbf{ExaWind: A Coupling of NaluWind with OpenFAST} \cite{Sprague:2019}\\
\begin{itemize}
\item NaluWind is an open-source massively parallel, incompressible flow solver for wind turbines and parks
\item Is a wind-focused adaption of the general flow solver NaluCFD
\item Focus on delivering flow simulations in exascale
\item Coupled to OpenFAST to perform multi-fidelity simulations of wind parks
\item Verified simulation on single turbine case against other codes and experiments
\item Emphasis on scalability and increased simulation speed through advanced solver and coupling algorithms and the optional use of GPUs
\item Extensive documentation with insights into the theory and verification 
\item How does it relate to our work
\begin{itemize}
\item Very close to what we want to do
\item Especially interesting for high-performance applications
\item Expert tool that needs some time to set up, understand, and run --> The hope is that our coupling to the "standard" tool OpenFOAM will have a lower barrier for beginners
\end{itemize}
\end{itemize}

\textbf{SOWFA} \cite{Churchfield:2012}
\begin{itemize}
\item Developed by NREL
\item Simulator for Wind Farm Applications based on OpenFOAM
\item Enables a coupling with OpenFAST
\item Methodology
\begin{itemize}
\item Implements a turbine class horizontalAxisALM in OpenFOAM to call OpenFAST and exchange data
\item This class can be included in any OpenFOAM solver
\item The solver calls OpenFAST during each timestep
\end{itemize}
\item How this relates to our idea
\begin{itemize}
\item Different approach: Is based on the modification of OpenFOAM source code while we strive to leave the solvers untouched
\item A lot of internal OpenFOAM knowledge necessary
\\
\end{itemize}
\end{itemize}

\textbf{foamFAST and MPCCI coupling tool} \cite{Weber:2017}
\begin{itemize}
\item Motivation
\begin{itemize}
\item Replace the low-fidelity aerodynamic calculations in FAST, based on the BEM theory, with high-fidelity calculations in OpenFOAM
\item BEM theory is not sufficient for many applications; CFD is needed
\item Use the coupling tool MpCCI to perform the coupling
\end{itemize}
\item Methodology
\begin{itemize}
\item Couple OpenFOAM and OpenFAST via the coupling tool MpCCI
\item MpCCI is a licensed coupling tool for partitioned multi-physics simulations developed by the Fraunhofer SCAI
\begin{itemize}
\item Connect tools to MpCCI via adapters
\item Once the adapter is written, you can couple your tool to any other tool connected to MpCCI
\item MpCCI acts as the master algorithm
\end{itemize}
\item This project developed an OpenFAST adapter from scratch and adapted an existing OpenFOAM adapter to perform the coupling
\item Source code is not violated
\end{itemize}
\item How it relates to our idea
\begin{itemize}
\item Very similar approach to preCICE´
\item Difference of MpCCI and preCICE: preCICE takes the library approach, while MpCCI takes the master algorithm approach
\item MpCCI is commercial and has limited HPC capabilities
\\
\end{itemize}
\end{itemize}

\begin{itemize}
\item Combines positive aspects of previous works while avoiding some drawbacks
\item Maintainability: Adapter is a separated piece of code which is easier to maintain than a modified OpenFOAM solver
\item Open access: No license
\item Plug and play: Adapter opens the road to connect OpenFAST not only to OpenFOAM, but to any CFD solver coupled to preCICE
\item Easy to use: preCICE creates a clean simulation environment, seperating the coupled tools clearly, from which students and professionals benefit
\end{itemize}
What are some serious drawbacks?
\begin{itemize}
\item There are currently no preCICE adapters for other CFD solvers that would be interesting to couple to: NaluWind, GRASP, ...
\item Developing such adapters would mean to re-implement a functionality that already exists in native coupling tools --> questionable if that is a good investment
\end{itemize}

\end{comment}

\section{Software description}
\label{section:software}

\subsection{Software components}

\subsubsection{preCICE}

preCICE \cite{Chourdakis:2022} is an open-source coupling library for partitioned multi-physics simulations. An overview of its concept is shown in Figure \ref{fig:precice:overview}. In preCICE terminology, the coupled simulation programs are called \textit{solvers} or \textit{participants}. preCICE connects different solvers to perform a partitioned simulation. It takes care of different coupling aspects such as data mapping and communication. The black-box approach ensures a flexible simulation setup \cite{Gatzhammer:2014} with many coupling options. Explicit and implicit coupling schemes are available, as well as multi-coupling to connect more than two participants. In addition, preCICE provides acceleration algorithms to reach a decent computation time for large simulations and time interpolation for the coupling of solvers with different time steps. The coupling scheme is defined in the \textit{precice-config.xml}, the only global file which is accessed by all participants.\\
For each solver, a specific \textit{adapter} is necessary to communicate with preCICE. The adapter is an additional software layer which can take many forms, depending on the solver. For example, the OpenFOAM adapter is a C++ function object, an independent library, while the FEniCS adapter is a Python module. Additional language bindings in Fortran, Julia and Matlab allow the development of adapters in these languages as well. preCICE is scalable and performs on personal laptops and HPC applications alike.

\begin{comment}

\begin{itemize}
	\item An overview is shown in Figure \ref{fig:precice:overview}
	\item Coupled simulation programs are called \textit{solvers} or \textit{participants}
	\item preCICE connects different solvers to perform a partitioned simulation
	\item preCICE takes care of different coupling aspects such as data mapping and communication
	\item Different coupling schemes can be implemented: Who is coupled o whom, what data is exchanged, which coupling algorithms are used \cite{Gatzhammer:2014}
	\item Multi-coupling: Couple multiple (in theory: arbitrary many) participants in different configurations (explicit, implicit)
	\item Acceleration: Use acceleration algorithms to reach a decent computation time
	\item Time interpolation: Use solvers with different time step sizes
	\item Coupling scheme is defined in the precice-config.xml, the only global file which is accessed by all participants
	\item For each solver, a specific \textit{adapter} is necessary to communicate to preCICE
	\item Adapter is an additional piece of software
	\item Adapter can take many forms depending on the solver, eg.: OpenFOAM - C++ Function object, FEniCS - Python module
	\item Different Language bindings are available: C/C++, Python, Fortran, Julia, Matlab
	\item The Adapter allows the solver to access preCICE and to call the coupling
\end{itemize}

\end{comment}

\begin{figure*}[h]
	\centering
	\includegraphics[width=0.9 \textwidth]{images/precice-overview.png}
	\caption{preCICE overview \cite{Chourdakis:2022}}
	\label{fig:precice:overview}
\end{figure*}


\subsubsection{OpenFOAM}

OpenFOAM\footnote{\url{https://www.openfoam.com/}}, an acronym for Open Field Operation and Manipulation, is a powerful open-source computational fluid dynamics (CFD) software package. This versatile tool provides a comprehensive suite of solvers and utilities for simulating and analyzing fluid flow, heat transfer, and related phenomena. Its open-source nature allows users to access, modify, and distribute the source code. Originally developed in the late 1980s, OpenFOAM has since evolved into a robust and widely-used simulation platform. It employs a finite volume method, making it particularly suitable for simulating complex fluid dynamics scenarios in various industries, including aerospace, automotive, energy, and environmental engineering.

A key strength of OpenFOAM lies in its flexibility and extensibility. Users can customize simulations by modifying the source code or by utilizing a wide range of pre-existing solvers and libraries. This adaptability makes OpenFOAM a preferred choice for researchers, engineers, and scientists seeking to tackle diverse fluid dynamics challenges. With a strong emphasis on parallel computing, OpenFOAM is capable of leveraging high-performance computing resources to accelerate simulations, enabling the analysis of large-scale and intricate fluid flow problems. Its user-friendly interface, coupled with extensive documentation and a supportive community, makes OpenFOAM an accessible tool for both beginners and experienced practitioners in the field of computational fluid dynamics.

\subsubsection{preCICE-OpenFOAM Adapter}

The preCICE-OpenFOAM adapter \cite{Chourdakis:2023} is designed to facilitate robust and efficient coupling between the preCICE library and OpenFOAM to enable multi-physics simulations. The software architecture of the adapter follows a modular design, allowing for flexibility and customization in the coupling process. Besides the core functionalities, the adapter consists of three distinct modules for fluid flow (FF), fluid-structure interaction (FSI), and conjugate heat transfer (CHT) coupling scenarios. Each module is optimized to handle the requirements of its respective physics domain.

The overall design emphasizes modularity, extensibility, and adherence to established standards, making the preCICE-OpenFOAM adapter an adaptable tool for a wide range of multi-physics simulations. More information on how to install, use, and modify the adapter can be found in the official documentation\footnote{\url{https://precice.org/adapter-openfoam-overview.html} (visited on 06.01.2024)}.

\begin{comment}
\begin{itemize}
\item OpenFOAM
\begin{itemize}
\item General information
\begin{itemize}
\item Leading open-source software for computational fluid dynamics
\item Used in academia and industry
\item Scalable from laptops to supercomputers
\item Use cases: aerodynamics, aeroacoustics, particles, chemical reactions, ...
\end{itemize}
\item Setup
\begin{itemize}
\item Build upon 
\end{itemize}
\item Enables the use and modification of different solvers
\item Used mainly for CFD but also FEM simulations 
\end{itemize}
\item Adapter
\begin{itemize}
\item Enables the coupling of OpenFOAM via preCICE in different simulation scenarios
\item Exemplifies the development strategy behind preCICE: Adapter is a library for OpenFOAM and called on runtime to perform the coupling without any modification to the OpenFOAM source code
\item Different modules for different kinds of multi-physics simulations: FSI, FF, CHT
\end{itemize}
\end{itemize}	
Note:
I should introduce the OpenFOAM adapter at some point.
Lets introduce the broad concept of the adapter here and add more details, eg about the specific files that would need to be modified, later in the Challenges part
\newline
\end{comment}



\subsubsection{OpenFAST}

OpenFAST\footnote{\url{https://openfast.readthedocs.io/en/main/}} is a versatile and open-source simulation tool developed for the analysis and design of wind turbines. This software, maintained by the National Renewable Energy Laboratory (NREL) in collaboration with the wind energy community, provides a comprehensive platform for simulating the aerodynamics, structural dynamics, and turbulence effects associated with wind turbine systems.

OpenFAST is designed to meet the evolving needs of the wind energy industry, offering a modular and extensible architecture that allows users to tailor simulations to specific requirements. Its capabilities span a range of applications, including load analysis, controller design, and overall turbine performance assessment. The software incorporates advanced modeling techniques for aerodynamics, structural dynamics, and control systems, ensuring accurate representation of wind turbine behavior in various operating conditions. With a focus on high-performance computing, OpenFAST can leverage parallel processing to handle complex simulations efficiently, facilitating the analysis of large-scale wind turbine systems. The tool supports the simulation of various turbine configurations, including land-based and offshore turbines with fixed or floating platforms. This flexibility makes it a valuable tool for addressing the diverse challenges associated with wind energy projects.

On a software level, OpenFAST is a glue code for different modules which compute the various physical domains. For example, the separate module AeroDyn is used to compute the aerodynamics of a simulation case. Another module called ElastoDyn computes the structural response of the aerodynamic loads. Both tools are then coupled in runtime in the framework. OpenFAST can be interfaced with a C++ API to drive the simulation and couple the simulation to an external CFD tool.

\begin{comment}

\begin{itemize}
	\item Wind energy engineering tool developed by NREL
	\item Widely used in academia and industry
	\item Simulates the coupled aerodynamic, structural, and electrical behaviour of wind turbines as well as the control response
	\item Allows to model on- and offshore wind turbines and can be used to compute wind parks as well
	\item OpenFAST is a glue code for different modules which compute the various physical domains
	\item Takes one main input file denoted \textit{.fst} which points to more specific files for each computation module (eg AeroDyn, ServoDyn, ...)
	\item Fidelity: Does OpenFAST include computations in different fidelities? Is it a low- to mid-fidelity tool?
	\item Computes the Fluid-Structure-Interaction at blades and tower with the actuator line method
	\item Inflow field is either computed by AeroDyn or received from external CFD solver
	\item Has a dedicated C++ API for the coupling with CFD solvers
\end{itemize}

\end{comment}

%OpenFAST is a low- to mid-fidelity tool mainly used to perform turbine loads analysis and drive the detailed turbine design. The computational intensity is therefore higher than for tools used in the early design exploration phase like WISDEM, RAFT or FLORIS, but lower than highly resolved solvers necessary to understand the actual physics and check the final turbine design like ExaWind or SOWFA. 

\subsection{Software concept}

The main idea of this work is to develop a preCICE-OpenFAST adapter, a new piece of software that connects OpenFAST and preCICE. It acts as a driver code towards OpenFAST, leveraging the C++ API of the simulation program. Simultanously, it calls preCICE to communicate data and receive steering commands for the simulation. The software is configurable on runtime by two \textit{.yaml} input files. \textit{preciceInput.yaml} contains information about the coupling setup such as which variables are exchanged, which meshes are used and where the global \textit{precice-config.xml} can be found. \textit{openfastInput.yaml} adds metadata for the wind turbine simulation and points towards the OpenFAST \textit{.fst} file.

\begin{figure}[b!]
	\centering
	\begin{minipage}{0.9\textwidth}
		\begin{minted}{cpp}
		#include <OpenFAST.H>
		#include <precice.hpp>
		
		vector<double> readData;
		vector<double> writeData;
		
		// OpenFAST Setup
		fast::OpenFAST FAST;
		...
		FAST.init()
		// preCICE Setup
		participant = precice.participant(...);
		...
		participant.initialize();
		// main time loop
		while participant.isCouplingOngoing(): 
			// Get read data from preCICE
			participant.readData(readData, ...);
			// Set read data in OpenFAST
			FAST.setVelocity(readData, ...);
			// Compute next time step
			FAST.step();
			// Get write data from OpenFAST
			FAST.getForce(writeData, ...);
			// Send write data to preCICE
			participant.writeData(writeData, ...);
			// Advance preCICE in time
			participant.advance(dt) 
		
		participant.finalize()
		FAST.end()
		\end{minted}
	\end{minipage}
	\caption{Concept of the OpenFAST adapter: The script utilizes the C++ API to execute OpenFAST and calls preCICE to couple the simulation. For conciseness, API calls are simplified.}
	\label{code:adapter}
\end{figure}

The software concept is presented in simplified form in Figure \ref{code:adapter}. First, the header files for important libraries are included and variables for data handling are created (lines 1-5). The script then utilizes the C++ API of OpenFAST and preCICE to instantiate and initialize both tools (lines 8-14). The \textit{FAST} object allows the access of OpenFAST while the \textit{participant} object is connected to preCICE. The coupling library controls the main time loop (line 16-28) which performs the coupled simulation. First, velocity data is retrieved from the CFD solver and stored locally (line 18). The information is then passed on to OpenFAST (line 20) and updated on the internal meshes of the simulation tool. OpenFAST is called to compute the wind turbine dynamics for the next time step (line 22). The resulting blade and tower force data is extracted and stored (line 24). The updated force data is passed on to the CFD solver (line 26), after which both solvers advance in time (line 28). This loop is repeated until the simulation time is completed. Finally, preCICE and OpenFAST are finalized (lines 30-31). The software is developed and documented on GitHub\footnote{\url{https://github.com/LeonardWilleke/openfast-adapter}}. 

\section{Example cases}
\label{section:cases}

To drive the adapter development and test the software at its current state, two example cases were implemented. The OpenFAST simulation of a single wind turbine, namely the NREL 5MW model \cite{Jonkman:2009}, should be coupled. The first case couples the simulation to a dummy fluid solver. While performing no realistic computations, the dummy is handy to get an insight into the different data structures, mesh setups, and mapping options. The second case couples OpenFAST to OpenFOAM. It can be seen as a first proof-of-principle with a successful data exchange between the tools. However, more work needs to be done to ensure a correct and robust simulation. Both cases are part of the repository on GitHub\footnote{\url{https://github.com/LeonardWilleke/openfast-adapter/tree/main/cases}}.

\begin{comment}
\begin{itemize}
\item Two example cases were implemented to drive the adapter development
\item The OpenFAST simulation of a single NREL 5MW turbine \cite{Jonkman:2009} should be coupled
\item Coupling to dummy fluid solver to test the data exchange
\item Coupling to OpenFOAM to test the mapping with a CFD tool
\item Both cases are part of the repository on GitHub\footnote{\url{https://github.com/LeonardWilleke/openfast-adapter/tree/main/cases}}\\
\end{itemize}
\end{comment}

\subsection{Coupling OpenFAST with a dummy CFD solver}

The reasoning behind this case is to provide a simple and fast way to test different functionalities and gain insight, not to perform realistic simulations. OpenFAST computes a NREL 5MW turbine with a fixed rotor to avoid mesh problems on the moving blades. The fluid solver is implemented as a simple C++ script that calls preCICE to enable the coupling. It creates a mesh that can be used to map data from OpenFAST on it and writes constant values back, but does not perform any calculations. The setup allows to explore the mapping and data exchange.

OpenFAST employs two internal meshes of the turbine surface with different vertices. Force data is stored on one mesh, while the flow velocity is stored on another. Both meshes can be accessed by the C++ API. However, it is also possible to exchange both variables with the velocity mesh and let OpenFAST map the force data to the force mesh internally. Inspired by the coupling setup in \cite{Taschner:2022}, I want to access both meshes and use preCICE for the mapping. This results in the elaborous coupling scheme seen in Figure \ref{fig:dummy:coupling}. In total, three meshes are employed: The Fluid solver has one mesh \textit{Fluid-Mesh} used for both the force and velocity data while the Solid solver (OpenFAST) uses two meshes. \textit{Solid-Mesh-Velocity} is the velocity mesh of OpenFAST, on which the velocity data from the Fluid solver is mapped. \textit{Solid-Mesh-Force} is the force mesh of OpenFAST, from which force data is mapped back to the Fluid solver.

\begin{figure*}[h]
	\centering
	\includegraphics[width=0.9\textwidth]{images/openfast-dummy-coupling-scheme.png}
	\caption{Coupling scheme between OpenFAST, named Solid, and the dummy Fluid solver. The figure was created from the \textit{precice-config.xml} using the \textit{config visualizer tool}\protect\footnotemark.}
	\label{fig:openfast:coupling}
\end{figure*}

\footnotetext{\url{https://precice.org/tooling-config-visualization.html}}

\begin{comment}
Case dummy-turbine
\begin{itemize}
\item Explain the setup
\begin{itemize}
\item Couple OpenFAST to a dummy fluid solver
\item OpenFAST computes a NREL 5MW turbine with a fixed rotor to avoid problems due to the moving mesh
\item Fluid solver is implemented as dummy: No CFD calculation takes place
\item The dummy creates a mesh on which data from OpenFAST is read and writes back constant values
\item Allows to explore the mapping and data exchange
\item Can be used for regression tests in the future (use in challenges)
\end{itemize}
\item Explain the mesh use of OpenFAST
\begin{itemize}
\item OpenFAST has two internal meshes of the turbine surface with different vertices
\item Force mesh: Used to store and compute the surface force
\item Velocity mesh: Used to store and compute the flow velocity
\item Possibility to let OpenFAST map between the two meshes
\item Both meshes can be accessed via the C++ API
\item We want to use preCICE for the mapping
\item A similar coupling setup is employed by \cite{Taschner:2022} who also uses both meshes
\item Maybe add a visualization of the coupling scheme to clarify the mesh use\\
\end{itemize}
\end{itemize}
\end{comment}



\subsection{Coupling OpenFAST with OpenFOAM}

The coupling of OpenFAST with OpenFOAM uses the same coupling scheme as presented in the previous case. However, we are dealing now with 

Case cfd-turbine
\begin{itemize}
	\item Explain the setup
	\begin{itemize}
		\item Couple OpenFAST to OpenFOAM
		\item The coupling scheme and mesh use is identical to the dummy
		\item Now we are dealing with a different mesh on the fluid solver
		\item Main obstacle: The OpenFOAM adapter is not designed to implement a coupling with a solver using the actuator line method. How to map between the line mesh in OpenFAST (Figure )\ref{fig:mesh:fast} and the volume mesh in OpenFOAM (Figure \ref{fig:mesh:foam})?
	\end{itemize}
	\item Explain how the current setup works
	\begin{itemize}
		\item Define a cellSet inside the OpenFOAM domain (Figure \ref{fig:mesh:foam}) to which OpenFAST is coupled
		\item Write the rotor and tower data to this subdomain
		\item Read velocity data from this subdomain
		\item The mapping from line to volume and back is done by preCICE, but probably wrong
	\end{itemize}
	\item How should the turbine be represented in OpenFOAM?
	\begin{itemize}
		\item Volume mesh / cellSet
		\item Surface mesh / patch
		\item ALM implementation (eg with turbinesFoam)\\
	\end{itemize}
\end{itemize}

\newpage
\begin{figure*}[h]
	\centering
	\begin{subfigure}{0.7\textwidth}
		\centering
		\includegraphics[width=\linewidth]{images/openfast-turbine-mesh.png}
		\caption{Line representation of the turbine in OpenFAST}
		\label{fig:mesh:fast}
	\end{subfigure}
	\vspace{2pt}
	\begin{subfigure}{0.7\textwidth}
		\centering
		\includegraphics[width=\linewidth]{images/openfoam-turbine-mesh.png}
		\caption{Volume representation of the flow field immediately around the turbine in OpenFOAM}
		\label{fig:mesh:foam}
	\end{subfigure}
	\caption{Mesh differences between OpenFAST and the CFD solver OpenFOAM. OpenFAST uses the Actuator Line Model to represent blades and towers in the flow field, while OpenFOAM calculates the whole flow field.}
	\label{fig:mesh}
\end{figure*}



\section{Challenges}
\label{section:challenges}

Current limitations of the adapter
\begin{itemize}
	\item Only for one turbine
	\item Only for onshore turbine
	\item Only proof of concept: Data is exchanged in a sensible way, no knowledge about the physical correctness\\
\end{itemize}

What are the next steps in the coupling of OpenFAST via preCICE?\\

Task 1: Mapping turbine data between actuator lines and volume meshes
\begin{itemize}
	\item Problem: The OpenFOAM adapter is not designed for the implementation of ALM
	\item We need to get velocity from a volume section --> possible
	\item We need to set force or pressure gradient on a volume section --> not possible
	\item Solution 1: Do the mapping in the OpenFAST adapter
	\begin{itemize}
		\item Something very similar has been done by AspFAST (LINK)
		\item Includes smearing of the actuator data which is good
		\item Possible Problem: I add volume data to the mesh in the OpenFAST adapter that the OpenFOAM adapter is not able to write to FOAM afterwards --> think about
	\end{itemize}
	\item Solution 2: Modify the OpenFOAM adapter
		\begin{itemize}
			\item Adapt the PressureGradient.C class of the FF module
			\item Include code to write pressure gradient on a cellset
			\item Take structure from Pressure.C class
			\item Problem: How to write the pressure gradient on a field in OpenFOAM?
			\item Pressure gradient seems to be the boundary condition of the velocity field U --> this is settable
			\item Requires the correct import of the velocity field and the correct retrieval of its boundary field --> give the commands you know
			\item Not sure if you can also set the pressure gradient on the pressure field itself but dont think so
			\item Open: How to do smearing if necessary\\
		\end{itemize}
\end{itemize}

Task 2: Bring the OpenFAST adapter to maturity
\begin{itemize}
	\item Write a regression test
	\item Create a first test case with documentation
	\item Verify simulation results against simulations done with AspFAST and other tools (\cite{Taschner:2022} gives some benchmark cases)\\
\end{itemize}

Task 3: Enable coupling with multiple turbines
\begin{itemize}
	\item OpenFAST C++ API allows to run multiple instances of OpenFAST for wind park scenarios
	\item How do you connect them to preCICE? 
	\item How do you define the different coupling scenarios?
	\item Possibly use the MacroMicro manager of preCICE to deal with the coupling of multiple domains\\
\end{itemize}

Task 4: Explore coupling scenarios with other CFD solver
\begin{itemize}
	\item Are there currently other CFD solvers coupled to preCICE that would be interesting?
	\item Otherwise interesting candidates are: YALES2, GRASP, NaluWind
	\item Most of them have native coupling tools for OpenFAST already\\
\end{itemize}

Additional remarks
\begin{itemize}
	\item The OpenFOAM library turbinesFoam\footnote{\url{https://github.com/turbinesFoam/turbinesFoam} (visited 14.12.2023)} \cite{Bachant:2018} might be useful. It implements the actuator line method with different solvers like pimpleFoam in OpenFOAM. The solvers are modified to perform the ALM computation. It is not clear yet how this could be of use, as we want to perform the ALM computation of the turbine to OpenFAST and map the results to OpenFOAM, not do the whole computation in OpenFOAM.
\end{itemize}


\section{Conclusion}

\begin{itemize}
	\item First coupling of OpenFAST and preCICE was presented
	\item Coupling with a dummy solver and OpenFOAM was discussed
	\item Although a proof of concept was achieved, some challenging tasks remain to enable a full coupling to CFD solvers
	\item How to map between OpenFAST and an arbitrary CFD solver? Where to place the mapping and smearing algorithm for the ALM method?
	\item This work may serve as a starting point
	\item Has the potential to be developed into a viable open-source alternative for the coupling of OpenFAST to different CFD solvers
\end{itemize}


\section*{Acknowledgments}

I am grateful to Prof. Axelle Viré who accepted me as a visiting researcher and made this work possible. Many thanks to her and Evert Wiegant for the discussions and hints during our meetings. Extended thanks to Benjamin Uekermann, who enabled my visit at TU Delft in the first place and provided feedback along the way. Lastly, I want to thank the wonderful crowd of PhD researchers in the wind energy department who made my time in Delft memorable.

%% Bibliography

\printbibliography

%% Appendix
\begin{comment}
\section*{Appendix}

Possible points to include in the Appendix
\begin{itemize}
\item Files from the OpenFOAM adapter with hints on how to modify them
\item Files from AspFAST with hints on how to reuse the code for our mapping
\end{itemize}
\end{comment}


\end{document}
