\section{Challenges}
\label{section:challenges}

What are the next steps in the coupling of OpenFAST via preCICE? This section aims to give an overview of the upcoming tasks to guide interested developers in their endeavour.

\subsection{Map data in the OpenFAST adapter}

\begin{comment}
\begin{itemize}
	\item preCICE and its adapters were not designed for the implementation of ALM
	\item mapping in preCICE does not include sufficient functions to translate between volume and line mesh
	\item We need sampling and smearing methods to connect them
	\item See aspast.cpp on how this could be done in the OpenFAST adapter
	\item Problems
	\begin{itemize}
		\item What to do with the internal preCICE mapping? Use only for data transfer? Feels weird and error-prone
		\item Which mesh to use in preCICE for FAST? The same fluid mesh as for FOAM. Coupling scheme needs to be changed
		\item Possibility to include such methods in preCICE internally? Unlikely
	\end{itemize}
\end{itemize}
\end{comment}

The most pressing issue is to implement a correct mapping. We face problems because preCICE and its components were not designed for the implementation of ALM and lack functions to translate data between volume and line meshes.
In particular, we need sampling and smearing methods. Sampling is used to choose a subset of data points from a volume mesh and map them on a line mesh. In our case, this means sampling velocity data from the CFD solver and mapping it to OpenFAST.
Smearing is used to distribute data from a line mesh over a volume mesh. Hence, we use it to take force data from OpenFAST and translate it to the CFD solver mesh. More insight into smearing for wind energy applications can be found in  \cite{Jha:2014} and \cite{Meyer:2019}.\\

The code\footnote{\url{https://gitlab.com/whiffle-public/aspfast} (visited on 28.12.2023)} from AspFAST provides a blueprint for sampling and smearing methods. The tool to connect OpenFAST and GRASP is published under a MIT license and can therefore be re-used without limitations. It implements useful mapping functions in the main script \textit{aspast.cpp}. To communicate from OpenFAST to the CFD solver, the functions \textit{calcBodyForce}, \textit{uniformBodyForce} and \textit{diskBodyForce} are used. To understand the mapping from CFD to OpenFAST, look into the functions \textit{sampleVelocity} and \textit{calcVelocity}. Including similar functions in the OpenFAST adapter could address the current mapping problems.\\

However, implementing additional mapping functions in the OpenFAST adapter raises questions on how to use the adapter with the preCICE core library. Using preCICE to communicate between solvers usually means to use the internal mapping facilities of preCICE, which would lead to double-mapping in our case. Also, the OpenFAST adapter will need access to the CFD volume mesh. Direct mesh access\footnote{\url{https://precice.org/couple-your-code-direct-access.html} (visited on 16.01.2024)} could solve both issued by circumventing the preCICE mapping and granting the OpenFAST adapter direct access to the CFD volume mesh stored in preCICE.

\subsection{Modify the OpenFOAM adapter}

Once the mapping issue is resolved, the adapter of the CFD solver will receive correct force values from the preCICE library. Now, these values have to be set in the CFD program. The adapter needs to access the internal force field of the CFD solver and overwrite it with the data obtained from the OpenFAST calculation.

The OpenFOAM adapter currently lacks this functionality. It has volume coupling functionalities, eg. to retrieve velocity data from OpenFOAM or write pressure data to it. It also has the capacity to exchange force data on surfaces. But writing force (or pressure gradient) data on a volume is currently not implemented. Note that this is a limitation of this specific adapter and how it interacts with the CFD solver, not of preCICE itself.\\

To solve this problem, the missing methods can be added to the OpenFOAM adapter. I propose to modify the Fluid-Fluid (FF) module of the adapter which is used in the current coupling setup. We need to adapt the class \textit{PressureGradient.C} of the FF module. The class implements methods to read and write pressure gradient data on surfaces (patches). A code section to write pressure gradient data on a volume (preferably a cellSet) should be added. Have a look in the class \textit{Pressure.C} of the FF module to get an idea how to access and write field values in OpenFOAM in general.

Inside OpenFOAM, the pressure gradient is stored as the boundary condition of the velocity field U. Therefore, we need to import the velocity field U in the class \textit{PressureGradient.C} and use appropriate commands from OpenFOAM to set its boundary condition. Setting the boundary condition on a cellSet and not on a wall or inlet might be tricky.

As an additional remark, the OpenFOAM library turbinesFoam\footnote{\url{https://github.com/turbinesFoam/turbinesFoam} (visited 14.12.2023)} \cite{Bachant:2018} might be useful. It implements the actuator line method with different solvers like pimpleFoam. The solvers are modified to perform the ALM computation. It is not clear yet how this could be of use, as we want to perform the ALM computation of the turbine in OpenFAST and map the results to OpenFOAM, not do the whole computation in OpenFOAM.

\begin{comment}

\begin{itemize}
	\item Problem: The OpenFOAM adapter is not designed for the implementation of ALM
	\item We need to get velocity from a volume section --> possible
	\item We need to set force or pressure gradient on a volume section --> not possible
	\item Solution 1: Do the mapping in the OpenFAST adapter
	\begin{itemize}
		\item Something very similar has been done by AspFAST (LINK)
		\item The MIT license of the software allows to re-use it without limitations
		\item What exactly needs to be done?
		\begin{itemize}
			\item Look into aspfast.cpp
			\item Understand how the mapping from OpenFAST to the CFD solver is done in the functions \textit{calcBodyForce}, \textit{uniformBodyForce} and \textit{diskBodyForce}
			\item Understand how the mapping from the CFD solver to OpenFAST works in the functions \textit{sampleVelocity} and \textit{calcVelocity}
		\end{itemize}
		\item Includes smearing of the actuator data which is good
		\item Possible Problem: I add volume data to the mesh in the OpenFAST adapter that the OpenFOAM adapter is not able to write to FOAM afterwards --> think about
	\end{itemize}
	\item Solution 2: Modify the OpenFOAM adapter
		\begin{itemize}
			\item Adapt the PressureGradient.C class of the FF module
			\item Include code to write pressure gradient on a cellset
			\item Take structure from Pressure.C class
			\item Problem: How to write the pressure gradient on a field in OpenFOAM?
			\item Pressure gradient seems to be the boundary condition of the velocity field U --> this is settable
			\item Requires the correct import of the velocity field and the correct retrieval of its boundary field --> give the commands you know
			\item Not sure if you can also set the pressure gradient on the pressure field itself but dont think so
			\item Open: How to do smearing if necessary\\
		\end{itemize}
	\item Additional remark: The OpenFOAM library turbinesFoam\footnote{\url{https://github.com/turbinesFoam/turbinesFoam} (visited 14.12.2023)} \cite{Bachant:2018} might be useful. It implements the actuator line method with different solvers like pimpleFoam in OpenFOAM. The solvers are modified to perform the ALM computation. It is not clear yet how this could be of use, as we want to perform the ALM computation of the turbine to OpenFAST and map the results to OpenFOAM, not do the whole computation in OpenFOAM.
\end{itemize}
\end{comment}

\subsection{Improve code maturity}

To railguard the further development, a regression test for the OpenFAST adapter should be implemented. The dummy fluid solver from chapter \ref{section:cases:dummy} can be used as a simple coupling participant to check calculations. The preCICE-FMI runner\footnote{\url{https://github.com/precice/fmi-runner/tree/main} (visited on 06.01.2024)} may serve as an example on how to write the test, place it in the GitHub repository and execute it automatically with GitHub workflows.

Once a mature coupling to OpenFOAM is achieved, the coupling should be verified. Previous work \cite{Taschner:2022} provides a benchmark case to cross-verify the coupling of a single NREL 5MW turbine with five other research LES codes.
\begin{comment}
\begin{itemize}
	\item Write a regression test (using the dummy fluid solver)
	\item Create a first test case with documentation
	\item Verify simulation results against simulations done with AspFAST and other tools (\cite{Taschner:2022} gives some benchmark cases)\\
\end{itemize}
\end{comment}

\subsection{Enable coupling with multiple turbines}

A future version of the adapter may include the coupling of multiple turbines in OpenFAST with one OpenFOAM instance. The OpenFAST C++ API allows to run multiple instances of OpenFAST to simulate wind park scenarios with FAST.FARM. But how do you communicate the results of multiple turbines to preCICE, and how do you organize the coupling to the CFD solver? This scenario increases the complexity in both adapters involved in the coupling setup. Again, a look into the AspFAST code could provide insight into how to deal with those challenges.

A different option may be the use of the preCICE Micro manager\footnote{\url{https://precice.org/tooling-micro-manager-overview.html} (visited on 06.01.2024)}. It is a tool to manage many micro simulations and couple them to one macro simulation. However, it must be said that the manager was not designed with large-scale FSI simulations in mind.
\begin{comment}
\begin{itemize}
	\item OpenFAST C++ API allows to run multiple instances of OpenFAST for wind park scenarios
	\item How do you connect them to preCICE? 
	\item How do you define the different coupling scenarios?
	\item Possibly use the MacroMicro manager of preCICE to deal with the coupling of multiple domains\\
\end{itemize}
\end{comment}


\subsection{Explore coupling scenarios with other CFD solver}

The current work is focused on the coupling of OpenFAST with OpenFOAM for the simple fact that it is the only suitable CFD solver in the preCICE ecosystem. The coupling of other tools like GRASP, NaluWind or YALES2 via preCICE might be interesting to create a simulation environment where CFD solvers could be swapped easily. Simulation results could be compared mor readily and the individual strengths of each program leveraged depending on the simulation case. However, this setup comes with the additional effort of developing preCICE adapters for the other CFD programs. As the tools above are already coupled to OpenFAST with mature, verified tools, this should only be done if it adds real benefit to the research community.

\begin{comment}
\begin{itemize}
\item Are there currently other CFD solvers coupled to preCICE that would be interesting?
\item Otherwise interesting candidates are: YALES2, GRASP, NaluWind
\item Most of them have native coupling tools for OpenFAST already\\
\end{itemize}

Additional remarks
\begin{itemize}
	\item The OpenFOAM library turbinesFoam\footnote{\url{https://github.com/turbinesFoam/turbinesFoam} (visited 14.12.2023)} \cite{Bachant:2018} might be useful. It implements the actuator line method with different solvers like pimpleFoam in OpenFOAM. The solvers are modified to perform the ALM computation. It is not clear yet how this could be of use, as we want to perform the ALM computation of the turbine to OpenFAST and map the results to OpenFOAM, not do the whole computation in OpenFOAM.
\end{itemize}
\end{comment}
